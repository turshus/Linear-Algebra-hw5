\documentclass[]{exam}

\usepackage{amsmath, amssymb}
\usepackage[margin=1in]{geometry}
\usepackage[en-US, showdow]{datetime2}
\usepackage{bm}
\usepackage{hyperref}


\title{Homework 5\\
	Due \DTMdate{2022-02-11} %Add a % at the begining of the line to remove the due date from the title (or just delete this line entirely) 
	}
\date{ %\today %You can remove the first % in this line to show the current days date.
	}
\author{%Your Name %You should probably change this to your actual name and remove the first %, at least if you want credit.
	}
\begin{document}
\maketitle

\printanswers %comment out this line to hide your answers.



\begin{questions}
		\question Consider the statements about symmetric matrices and indicate if the statements are true or false. If a statement is true provide a proof, otherwise give a counterexample. 
		
		\begin{parts}
			\part The block matrix $\begin{bmatrix} 0 & A\\ A & 0 \end{bmatrix}$ is automatically symmetric.
			\begin{solution}
				You can write your solution here!
			\end{solution}
			\part If $A$ and $B$ are symmetric then $AB$ is symmetric.
			\begin{solution}
				You can write your solution here!
			\end{solution}
			\part If $A$ is not symmetric, then $A^{-1}$ is not symmetric.
			\begin{solution}
				You can write your solution here!
			\end{solution}
			\part If $A$, $B$, and $C$ are symmetric, then $(ABC)^T=CBA$
			\begin{solution}
				You can write your solution here!
			\end{solution}
		\end{parts}
		
		\question \S 3.1 \# 10. Which of the following subsets of $\mathbb{R}^3$ are actually subspaces? You should either prove that the set is a subspace or show that the set does not have one of the properties of subspaces.
	
	\begin{parts}
		\part The plane of vectors $(b_1, b_2, b_3)$ with $b_1 = b_2$.
		\begin{solution}
			We can re-write the plane of vectors to more accurately represent our constraints:
            \[
                \begin{bmatrix}
                    b_1 & b_1 & b_2
                \end{bmatrix}
            \]
            This is a subset of $\mathbb{R}^3$ and we can prove it by proving that the sum of any two vectors within the subsetis also in the subset:
            \[
                \begin{bmatrix}
                    b_1 \\ b_1 \\ b_2
                \end{bmatrix} +
                \begin{bmatrix}
                    b_3 \\ b_3 \\ b_4
                \end{bmatrix} =
                \begin{bmatrix}
                    b_1 + b_3 \\
                    b_1 + b_3 \\
                    b_2 + b_4
                \end{bmatrix}
            \]
            As we can see, the first two elemetns of the vectors equal eqch other which satisfies this constraint.  We also need to prove that any scalar multiple of a vector in the subset is also in the subset.
            \[
                \alpha \begin{bmatrix}
                    b_1 \\ b_1 \\ b_2
                \end{bmatrix} +
                \beta \begin{bmatrix}
                    b_3 \\ b_3 \\ b_4
                \end{bmatrix}
            \]
            \[
                \begin{bmatrix}
                    \alpha \cdot b_1 \\
                    \alpha \cdot b_1 \\
                    \alpha \cdot b_2
                \end{bmatrix} +
                \begin{bmatrix}
                    \beta \cdot b_3 \\
                    \beta \cdot b_3 \\
                    \beta \cdot b_4
                \end{bmatrix}
            \]
		\end{solution}
		\part The plane of vectors with $b_1 = 1$.
		\begin{solution}
			You can write your solution here!
		\end{solution}
		\part The vectors $(b_1, b_2, b_3)$ with $b_1b_2b_3 = 0$.
		\begin{solution}
			You can write your solution here!
		\end{solution}
		\part All linear combinations of $\vec{v} = (1,4,0)$ and $\vec{w} = (2,2,2)$.
		\begin{solution}
			You can write your solution here!
		\end{solution}
		\part All vectors $(b_1, b_2, b_3)$ with $b_1 + b_2 + b_3 = 0$.
		\begin{solution}
			You can write your solution here!
		\end{solution}
		\part All vectors $(b_1, b_2, b_3)$ with $b_1 \leq b_2 \leq b_3$.
		\begin{solution}
			You can write your solution here!
		\end{solution}
	\end{parts}

	\question An exercise using the outer-product method of matrix multiplication. Every matrix with rank $r$ can be written as the sum of $r$ rank 1 matices. An easy way to write a rank 1 matrix is using an outer-product (recall: $\vec{u}\vec{v}^T$ is an outer-product). Construct a matrix $A$ with rank 2 that has $C(A) = span((1,2,4), (2,2,1))$ and $C(A^T) = span((1,0),(1,1))$, you should use outer-products to find $A$. Then find a factorization on $A$ into a 3 by 2 matrix times a 2 by 2 matrix, you should think backwards about the outer product method of matrix multiplication to help you.

\begin{solution}
	You can write your solution here!
\end{solution}

\question Suppose that $A$ is a $m \times n$ matrix and $\vec{b}$ is a $m \times 1$ vector. Let $B$ be the $m \times (n+1)$ matrix formed by adding $\vec{b}$ to $A$, so $B = [A \, \vec{b}]$. What must be true so that $C(A)=C(B)$? What must be true if $C(A)$ is smaller that $C(B)$? Explain what must be true for $A \vec{x} = \vec{b}$ and $B \vec{x} = \vec{b}$ to have solutions. 


\begin{solution}
	In order for $C(A) = C(B)$, $\vec{b}$ must be a linear combination of some column in matrix $A$.  In order for $C(A) < C(B)$, $\vec{b}$ must be linearly independent of any columns in matrix $A$.  In order for $A\vec{x}=\vec{b}$ \emph{and} $B\vec{x}=\vec{b}$ to both have solutions, $\vec{b}$ simply has to be a linear combination of some column in its corresponding matrix.  For example:
	\[
		A = \begin{bmatrix}
			a & b & c \\
			d & e & f \\
			g & h & i
		\end{bmatrix}
		\begin{bmatrix}
			0 \\ 0 \\ 1
		\end{bmatrix} =
		\begin{bmatrix}
			c \\ f \\ i
		\end{bmatrix}
	\]
	\[
		B = \begin{bmatrix}
			a & b & c & j \\
			d & e & f & k \\
			g & h & i & l
		\end{bmatrix}
		\begin{bmatrix}
			0 \\ 0 \\ 0 \\ 1
		\end{bmatrix} =
		\begin{bmatrix}
			j \\ k \\ l
		\end{bmatrix}
	\]
\end{solution}

\newpage
\question Suppose that $A = 
\begin{bmatrix}
	1 & 2 & 0 & 1\\
	-1 & -2 & 1 & 0\\
	2 & 4 & 0 & 2
\end{bmatrix}$
\begin{parts}
	

	\part Describe the column space of $A$ by listing a basis for it.
		
	\begin{solution}
		The column space can be found by finding the pivots after elemination, so let's elimniate!
		\[
			EA = 
			\begin{bmatrix}
				1 & 0 & 0 \\
				1 & 1 & 0 \\
				-2 & 0 & 1
			\end{bmatrix}
			\begin{bmatrix}
				1 & 2 & 0 & 1 \\
				-1 & -2 & 1 & 0 \\
				2 & 4 & 0 & 2
			\end{bmatrix} =
			\begin{bmatrix}
				1 & 2 & 0 & 1 \\
				0 & 0 & 1 & 1 \\
				0 & 0 & 0 & 0
			\end{bmatrix}
		\]
		It's clear to see now that our pivots are in $x_1$ and $x_3$.  The basis for the column space $A$ can be described by the columns that contain the pivots.
		\[
			C(A) = span\left(\left\{
				\begin{bmatrix}
					1 \\ -1 \\ 2
				\end{bmatrix} ,
				\begin{bmatrix}
					0 \\ 1 \\ 0
				\end{bmatrix}
			\right\}\right)	
		\]
	\end{solution}
	
	\part Describe the nullspace of $A$ by listing a basis for it.
	
	\begin{solution}
		We can start to find the nullspace by referencing our elemination matrix $E$ that we solved above.
		\[
			EA = 
			\begin{bmatrix}
				1 & 0 & 0 \\
				1 & 1 & 0 \\
				-2 & 0 & 1
			\end{bmatrix}
			\begin{bmatrix}
				1 & 2 & 0 & 1 \\
				-1 & -2 & 1 & 0 \\
				2 & 4 & 0 & 2
			\end{bmatrix} =
			\begin{bmatrix}
				1 & 2 & 0 & 1 \\
				0 & 0 & 1 & 1 \\
				0 & 0 & 0 & 0
			\end{bmatrix}
		\]
		We can see by looking at the matrix $EA$ that we have pivots at $x_1$ and $x_3$ which means that our free variables are $x_2$ and $x_4$.  Let's set $x_2 = 1$ and $x_4 = 0$.
		\[
			\begin{bmatrix}
				1 & 2 & 0 & 1 \\
				0 & 0 & 1 & 1 \\
				0 & 0 & 0 & 0
			\end{bmatrix} =
			\begin{array}{r}
				x_1 + 2x_2 + x_4 = 0 \\
				x_3 + x_4 = 0
			\end{array}
		\]
		\[
			\begin{array}{c}
				x_1 + 2(1) + 0 = 0 \rightarrow x_1 = -2	\\
				x_3 + 0 = 0 \rightarrow x_3 = 0
			\end{array}
		\]
		We now have one solution. We can switch the values of $x_2$ and $x_4$ to get our 2nd solution.
		\[
			\begin{array}{c}
				x_1 + 2(0) + 1 = 0 \rightarrow x_1 = -1	\\
				x_3 + 1 = 0 \rightarrow x_3 = -1
			\end{array}
		\]
		We can now describe the nullspace with the basis for it:
		\[
			N(A) = span\left(\left\{
				\begin{bmatrix}
					-1 \\ 0 \\ -1 \\1
				\end{bmatrix},\begin{bmatrix}
					-2 \\ 1 \\ 0 \\ 0
			\end{bmatrix}
			\right\}\right)
		\]
	\end{solution}
	\newpage
	\part Describe the left nullspace of $A$ by listing a basis for it.
	
	\begin{solution}
		We can find the left nullspace of $A$ by referencing the elemination we did on $A$ in the problem above.
		\[
			EA = 
			\begin{bmatrix}
				1 & 0 & 0 \\
				1 & 1 & 0 \\
				-2 & 0 & 1
			\end{bmatrix}
			\begin{bmatrix}
				1 & 2 & 0 & 1 \\
				-1 & -2 & 1 & 0 \\
				2 & 4 & 0 & 2
			\end{bmatrix} =
			\begin{bmatrix}
				1 & 2 & 0 & 1 \\
				0 & 0 & 1 & 1 \\
				0 & 0 & 0 & 0
			\end{bmatrix}
		\]
		Pay attention to row 3 in $EA$.  It is all 0's.  Because the left nullspace is just the transpose of the nullspace, we can think of the rows as columns and vice versa.  This would mean that row 3 in $EA$ would actually be column 3 in $EA^T$.  This would mean that the third row in our elemination matrix $E$ would be our only solution to the left nullspace.
		\[
			N(A)^T	= span \left(\left\{
				\begin{bmatrix}
					-2 \\ 0 \\ 1
				\end{bmatrix}
				\right\}\right)
		\]
	\end{solution}
	
	\part Describe the row space of $A$ by listing a basis for it.
	
	\begin{solution}
		Much like finding the column space, we can start by finding $EA$:
		\[
			EA = 
			\begin{bmatrix}
				1 & 0 & 0 \\
				1 & 1 & 0 \\
				-2 & 0 & 1
			\end{bmatrix}
			\begin{bmatrix}
				1 & 2 & 0 & 1 \\
				-1 & -2 & 1 & 0 \\
				2 & 4 & 0 & 2
			\end{bmatrix} =
			\begin{bmatrix}
				1 & 2 & 0 & 1 \\
				0 & 0 & 1 & 1 \\
				0 & 0 & 0 & 0
			\end{bmatrix}
		\]
		Now we simply pay attention to where the pivots are located in the rows, and those are our answers.  The pivots are located at $x_1$ and $x_3$.
		\[
			C(A)^T = span \left(\left\{
				\begin{bmatrix}
					1 & 2 & 0 & 1
				\end{bmatrix},
				\begin{bmatrix}
					0 & 0 & 1 & 1
				\end{bmatrix}
			\right\}\right)	
		\]
	\end{solution}
\end{parts}


\end{questions}



\end{document}